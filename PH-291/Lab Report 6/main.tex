\NeedsTeXFormat{LaTeX2e}

\documentclass[12pt]{article}
\usepackage[letterpaper,portrait, margin=1in]{geometry}
\usepackage{booktabs, pgfplots, bm, multirow, amsmath, wrapfig, gensymb, graphicx}
\pgfplotsset{width=11cm,compat=1.9}
\renewcommand{\arraystretch}{1.5}
\usepackage[colorlinks=true, allcolors=blue]{hyperref}
\usepackage{indentfirst}
\usepackage{caption}



\begin{document}

%%%%%%%%%%%%%%%%%%%%%%%%%%%%%%%%%%%%%%%%%%%%%%%%%%%%%%%%%%%%
%%% COVER PAGE - TO BE COPIED AT BEGINNING OF LAB REPORT %%%
%%%      PLEASE ALSO FILL OUT WITH YOUR INFORMATION      %%%
%%%%%%%%%%%%%%%%%%%%%%%%%%%%%%%%%%%%%%%%%%%%%%%%%%%%%%%%%%%%

\begin{titlepage}

       %\vspace*{.5cm}
        \begin{center}
        \textbf{\huge PH-291 Physics Lab} \\ 
        \textbf{\Large Professor Corn-Agostini} \\ 
        \textbf{\Large Fall 2022} \\ 
        \vspace*{.5cm}  
        \textbf{\large Lab \# 6: Python Plotting}
        \vspace{0.5cm}
        \end{center}
       
\noindent Your Name: Perla Berkovitz\\ \\
\noindent Your Lab Section: E\\ \\
\noindent Your Lab Instructor: Professor Corn-Agostini\\ \\
\noindent Your Lab Partner's Name: Sharon Sitt\\ \\
\noindent Read and sign Academic Integrity Statement:\\

\noindent {\em I hereby attest that I have not given or received any unauthorized assistance on this assignment.}

    \begin{center}
    \line(1,0){300} \\
    Sign here
    \end{center}

\noindent\textbf{\large Grading Rubric} \\ \\
\renewcommand{\arraystretch}{1.5}
%\large
\begin{tabular}{|l|c|r|l|} \hline
 {\bf CATEGORY \hspace{1cm}} & {\bf POINTS} & {\bf GRADE}\\\hline 
Purpose & 1 & \\\hline
Double Slit Plot & 3 & \\\hline
N-Slit Plot & 3 & \\\hline 
Single-Slit Plots & 6 & \\\hline 
Question & 1 & \\\hline 
Conclusion  & 1 & \\\hline \hline 
{\em Total} & 15 & \\ \hline
\end{tabular}

\end{titlepage} 



\newpage
\tableofcontents
\newpage
\section{Purpose}
In this lab, Python is used to plot the single, double, and N-slit interference and diffraction patterns across a range of $\theta$ values.
The peaks of various wavelengths and slit widths can then be more easily compared.
The first part has the Young's Double Slit Interference plotted with intensity $(I)$ on the y-axis and $\theta$ on the x-axis. 
This plot assumes infinitesimally small slit widths, which is changed to provide results that align more with real-world experiments in the final plots.
A plot for the N-Slit Interference pattern is then plotted with three different values for N. 
The final two plots use the fixed slit width, with the first plot showing the pattern formed by a single slit 1.5 micrometers wide with both red and blue light.
The second of these plots shows the pattern formed by a Youngs Double Slit with slits of finite width.
Each of these plots provides a graphical representation of the derived equations for the interference patterns.
\newpage
\section{Results}
\begin{center}
\includegraphics*[]{yds.png}\\
Plot 1: Young's Double Slit Interference\\
\end{center}
Plot 1 uses the \verb+np.linspace+ command:
\begin{verbatim}
    theta = np.linspace(-np.pi/2.0, np.pi/2.0, 1000)
\end{verbatim}
The slit separation here is given to be exactly 1 $\mu m$, and the wavelength used is that of red light (650 $nm$).
Here, it is assumed that the slits have infinitesimally small width, which is not necessarily true in real life experiments. 
This is accounted for in Plot 4.
\begin{center}
    \includegraphics*[]{ns.png}\\
    Plot 2: N-Slit Interference\\
\end{center}
This plot shows the interference of N-slits. Lines were plotted for 3 different values of N to show the impact of the amount of slits on the interference.
The equation of the line is \[I=I_0 \left[\frac{\sin N\delta/2}{\sin\delta/2}\right]^2 \quad \delta=\frac{2\pi}{\lambda}d\sin\theta\]
Here, $I_0=\frac{1}{N^2}$, and $d = 1\: \mu m$. The wavelength ($\lambda$) and \verb+np.linspace+ command are the same as in Plot 1. 
\begin{center}
    \includegraphics*[]{ssd.png}\\
    Plot 3: Single Slit Diffraction\\
\end{center}
Plot 3 shows the pattern formed by a slit with $a=1.5\:\mu m$. Each of the lines represents the diffraction due to a red and blue light respectively.
This curve is then used in the next plot to determine the interference a fixed slit.
\newpage
\begin{center}
    \includegraphics*[]{ydsi.png}\\
    Plot 4: Young's Double Slit Interference\\ 
\end{center}
Plot 4 combines the interference and diffraction to provide the interference of Young's Double Slit for a finite slit width. 
This red curve is the product of the curve graphed in Plot 1 and the curve plotted in Plot 3.
The slit width $a$ is 2 times the wavelength, which here was the wavelength of red light. 
The slit separation here was 3 times the slit width, but this may vary in future plotting. 
\newpage
\section{Conclusion}
Combining the graphs for Young's Double Slit diffraction and interference, a plot for the intensity was created. 
For the N-slit interference plot, python allows us to see that irrespective of the number of slits, the interference pattern will have the same values of the primary maxima.
Using python, it becomes easier to understand complex equations using their graphic interpretations. 
Python allows users to change the values of variables in equations and plot them alongside each other so users may understand how changes impact their final calculated results.
\newpage
\section{Question}
\subsection*{Why is the $\bm{m=3}$ order missing in Plot 4?}
The $m=3$ peak is missing because it lies at the same point as the minimum of the Single Slit Diffraction curve.
Generally, missing orders occur when a diffraction minimum occurs at the same value as the interference maximum.
In our case, this occurs at the $m=3$ order. 
The other orders displayed on the graph don't have this issue, making $m=3$ the only missing order.

\end{document}