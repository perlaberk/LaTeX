\begin{center}
        \begin{tabular}{|l c|}
        \hline
        \textbf{Approximate Focal Length: } & 5.50 cm \\
        \textbf{Instrumental Error:} & 0.05 cm \\
        \textbf{Approximate Focal Length:} & $5.50\pm0.05$ cm \\ \hline        
        \end{tabular}
        \vspace{3mm}
        \\Table 1: Approximate Focal Length of Lens F1
        \vspace{5mm}\\
        \begin{tabular}{|l c|}
            \hline
            \textbf{Approximate Focal Length: } & 10.00 cm \\
            \textbf{Instrumental Error:} & 0.05 cm \\
            \textbf{Approximate Focal Length:} & $10.00\pm0.05$ cm \\ \hline        
        \end{tabular}
        \vspace{3mm}
        \\Table 2: Approximate Focal Length of Lens C\\
        %\vspace{3mm}
    \end{center}
These approximate focal lengths were measured using a meter stick. 
The distance from the source to the lens was taken to be infinite, which allowed us to use the thin lens equation to approximate  our focal lengths. 
Multiple measurements were not taken because the focal lengths were meant to be approximate values.
For future work, multiple measurements will be taken to reduce the inherent random error present when taking measurements.
Using the approximate focal lengths of each lens, the optical bench was then set up. 
The distance between the source and the screen $(D)$ must be 4 times greater than the focal length.\\
\begin{center}
    \begin{tabular}{|c|c|c|c|c|}
        \hline
    \textbf{Measurement} & \textbf{Position 1 (cm)} & \textbf{Position 2 (cm)} & $\bm{D}$\textbf{ (cm)} & $\bm{d}$\textbf{ (cm)} \\ \hline
        1 & 14.00 & 7.85 & 21.70 & 6.15\\ 
        2 & 13.80 & 7.70 & 21.70 & 6.10\\ 
        3 & 14.00 & 8.10 & 21.70 & 5.90\\  \hline
        \multicolumn{5}{|c|}{\textbf{Instrumental Error:} 0.05 cm} \\
        \multicolumn{5}{|c|}{\textbf{Random Error $\bm{(d)}$:} 0.13 cm} \\
        \hline
    \end{tabular}
    \vspace{3mm}
    \\Table 3: Optical Bench Data for Lens F1\\
    \vspace{5mm}
\newpage
    \begin{tabular}{|c|c|c|c|c|}
        \hline
        \textbf{Measurement} & \textbf{Position 1 (cm)} & \textbf{Position 2 (cm)} & $\bm{D}$\textbf{ (cm)} & $\bm{d}$\textbf{ (cm)} \\ \hline
        1 & 27.10 & 17.30 & 41.70 & 9.80\\ 
        2 & 26.80 & 16.80 & 41.70 & 10.00\\ 
        3 & 26.40 & 16.30 & 41.70 & 10.10\\  \hline
        \multicolumn{5}{|c|}{\textbf{Instrumental Error:} 0.05 cm} \\
        \multicolumn{5}{|c|}{\textbf{Random Error $\bm{(d)}$:} 0.15 cm} \\
        \hline
    \end{tabular}
    \vspace{3mm}
    \\Table 4: Optical Bench Data for Lens C\\
    \vspace{10mm}
    \begin{tabular}{|c|c|c|}
        \hline
        \textbf{Line 1 Length (cm)} & \textbf{Line 2 Length (cm)} & \textbf{Angular Magnification} \\ \hline
        20.00 & 9.50 & 2.11 \\ \hline
        \multicolumn{3}{|c|}{\textbf{Instrumental Error:} 0.05 cm} \\ 
        \multicolumn{3}{|c|}{\textbf{Experimental Angular Magnification} 2.11$\pm$0.05} \\ \hline
    \end{tabular}
    \vspace{3mm}
    \\Table 5: Experimental Angular Magnification\\
\end{center}
The calculated angular magnification value was compared to the experimental magnification factor to confirm that the telescope functioned as expected.
This led us to assume that only one set of measurements was needed. For future testing, more measurements of the line lengths will be taken to decrease the inherent random error.