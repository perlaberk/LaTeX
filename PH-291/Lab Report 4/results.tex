\begin{center}
    
    \centering
    \begin{tabular}{|c|c|c|c|}
    \hline
        $f$ & $\frac{\partial f}{\partial d}$ & $\frac{\partial f}{\partial D}$ & Error of $f$  \\ \hline
        5.00 & -0.14 & 0.27 & 0.03 \\ \hline
        \multicolumn{4}{|c|}{$\bm{f: 5.00\pm0.03}$}\\\hline
    \end{tabular} 
    \vspace{3mm}
    \\ Table 6: $f$ Final Value for Lens F1 by Bessel's Method\\
    \vspace{6mm} 
    \centering
    \begin{tabular}{|c|c|c|c|}
    \hline
        $f$ & $\frac{\partial f}{\partial d}$ & $\frac{\partial f}{\partial D}$ & Error of $f$  \\ \hline
        9.83 & -0.12 & 0.26 & 0.03 \\ \hline
        \multicolumn{4}{|c|}{$\bm{f: 9.83\pm0.03}$}\\\hline
    \end{tabular} 
    \vspace{3mm}
    \\ Table 7: $f$ Final Value for Lens C by Bessel's Method\\
    \vspace{6mm}
    \centering
    \begin{tabular}{|c|c|c|c|}
    \hline
        $m_\theta$ & $\frac{\partial m_\theta}{\partial f_{obj}}$ & $\frac{\partial m_\theta}{\partial f_{eye}}$ & Error of $m_\theta$  \\ \hline
        -1.965 & -0.200 & 0.393 & 0.014 \\ \hline
        \multicolumn{4}{|c|}{$\bm{m_\theta: -1.965\pm0.014}$}\\\hline
    \end{tabular} 
    \vspace{3mm}
    \\ Table 8: $m_\theta$ Final Value \\
    \vspace{6mm}
\end{center}
\indent \indent The approximate focal length of lens F1 was $5.50\pm0.05$ cm, which does not lie within the same uncertainty range as the calculated focal length for lens F1 of $5.00\pm0.13$ cm. 
This could be attributed to the approximation process, which takes the distance between the source and the lens to be infinite, while it instead was a large finite value (discussed further in questions section).
Additionally, the approximation required the light source to be directly above the lens and the paper to lie flat on the floor. 
The measurements for approximate focal lengths may vary if the light or the paper is not properly positioned.
Another significant source of error arose due to the need for human perception. 
Depending on the experimenter's eyesight, they may see the different levels of focus compared to their partner.
This would produce variation in the focal length measurements and calculations depending on which experimenter viewed the image to determine the point at which the image would be sharpest.
In setting up the optical bench, the source, lens, and screen all needed to be aligned vertically. This also provides a source of error, in that if they are not properly aligned, the calculated focal lengths would be different.\\
\indent
The telescope construction involved placing the lenses into clips onto a meter stick. 
These clips fit snugly on the meter stick, but the lenses required tape to hold them in place.
Additionally, the telescope should have been held straight to ensure vertical alignment of the lenses, but due to the length of the meter stick, it was somewhat challenging to keep everything in line.
The lines were drawn based on the calculated magnification but were not exact due to difficulty getting perfect lines with the chalk. 
This gave an experimental magnification of $2.11\pm0.05$, as opposed to the theoretical magnification of $1.965\pm0.014$. 
The negative sign reported in Table 8 reflects the inversion of the final image seen by the observer.
The experimental and theoretical magnification values do not lie within the same error range, which can be attributed to the sources of error discussed above. 
If the lines had been redrawn to collect additional measurements, the experimental and theoretical values would be more similar. 
