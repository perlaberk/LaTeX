\begin{center}
    
    \centering
    \begin{tabular}{|c|c|c|}
        \hline
         & $\bm{x_{+1}}$ & $\bm{x_{-1}}$ \\ \hline
        $\bm{\theta}$ \textbf{(rad)} & 0.738 & 0.726 \\ 
        $\bm{\delta\theta}$ \textbf{(rad)} & 0.006 & 0.005 \\
        $\bm{\lambda}$ \textbf{(cm)} & $6.727\times10^{-5}$ & $6.641\times10^{-5}$ \\ 
        $\bm{\delta\lambda}$\textbf{ (cm)} & $4\times10^{-7}$ & $4\times10^{-7}$\\\hline
        \multicolumn{3}{|c|}{Experimental Wavelength: $\bm{(6.68\pm0.04)\times10^{-5}}$ \textbf{cm}}\\ \hline
    \end{tabular} 
    \vspace{3mm}
    \\ Table 6: Wavelength determination with grating\\
\end{center}
Here, $\delta\theta$ and $\delta\lambda$ are the error values for $\theta$ and $\lambda$ respectively, determined via error propagation. 
The experimental wavelength error reported is half the range of the two calculated wavelengths.  
Since the experimental error is the average of the two propagated errors, we can assume for future calculations that the SDOM of the measurements will be sufficient to provide the uncertainty associated with the final calculated value.
\begin{center}
    \vspace{6mm} 
    \centering
    \begin{tabular}{|c|c|c|c|c|}
        \hline
        & $\bm{x_{+1}}$ &$\bm{x_{-1}}$ & $\bm{x_{+2}}$ & $\bm{x_{-2}}$  \\ \hline
        $\bm{\theta}$ \textbf{(rad)} & 0.003 & 0.003 & 0.006 & 0.007 \\
        $\bm{\lambda}$ \textbf{(cm)} & $5.694\times10^{-5}$ & $6.383\times10^{-5}$ & $6.363\times10^{-5}$ & $6.839\times10^{-5}$ \\ \hline
    \end{tabular} 
    \vspace{3mm}
    \\ Table 7: Wavelength Determination with 0.2 mm Slit Width\\
    \vspace{5mm}
    \centering
    \begin{tabular}{|c|c|c|c|c|}
        \hline
        & $\bm{x_{+1}}$ &$\bm{x_{-1}}$ & $\bm{x_{+2}}$ & $\bm{x_{-2}}$  \\ \hline
        $\bm{\theta}$ \textbf{(rad)} & 0.002 & 0.002 & 0.004 & 0.004 \\
        $\bm{\lambda}$ \textbf{(cm)} & $4.552\times10^{-5}$ & $5.549\times10^{-5}$ & $5.560\times10^{-5}$ & $5.452\times10^{-5}$ \\ \hline
    \end{tabular} 
    \vspace{3mm}
    \\ Table 8: Wavelength Determination with 0.3 mm Slit Width\\
    \vspace{5mm}
    \begin{tabular}{|c|c|c|c|c|}
        \hline
        & $\bm{x_{+1}}$ &$\bm{x_{-1}}$ & $\bm{x_{+2}}$ & $\bm{x_{-2}}$  \\ \hline
        $\bm{\theta}$ \textbf{(rad)} & 0.002 & 0.002 & 0.004 & 0.004 \\
        $\bm{\lambda}$ \textbf{(cm)} & $8.008\times10^{-5}$ & $6.640\times10^{-5}$ & $7.232\times10^{-5}$ & $7.563\times10^{-5}$ \\ \hline
    \end{tabular} 
    \vspace{3mm}
    \\ Table 9: Wavelength Determination with 0.4 mm Slit Width\\
    \vspace{5mm}
    $\boxed{\bm{\lambda=(6.3\pm 0.3)\times10^{-5}}\textbf{ cm}}$
    \vspace{3mm}
    \\Final Value of $\bm{\lambda}$\\
    \vspace{5mm}
    \includegraphics*[]{Wavelength Values.jpg}
    \\Plot 1: Mean Wavelength Values with Associated Uncertainty
\end{center}
Plot 1 gives the final values of the wavelength as determined in parts A and B alongside the accepted value. 
Although the wavelength of Part B has a larger uncertainty than Part A ($0.3\times10^{-5}$ cm as opposed to $0.04\times10^{-5}$ cm), both wavelength values lie within the range of accepted values for a red laser.
The mean values of Part A and Part B differed by $3.65\times10^{-6}$ cm. The accepted value lies almost exactly in between the estimates of Part A and B, with Part A being off by $1.85\times10^{-6}$ cm, and Part B off by $1.80\times10^{-6}$ cm.
The mean of Part A and Part B is equivalent to the average accepted value for the wavelength of a red laser beam of $6.50\times10^{-5}$ cm.
For both part A and B, error was introduced when measuring the distance between the central maximum and outer maxima. The maxima were marked using a mechanical pencil with a lead thickness of 0.38 mm. 
While this is relatively thin compared to most other pencils, it was still challenging to mark down the maxima. 
Part B of the experiment used the minima, rather than the maxima, which required marking the edges of the maxima and finding the center. 
As the single slit became wider, this became more challenging as the maxima grew closer to each other. 
The midpoint of the maxima was estimated and then marked using the same mechanical pencil.
The distance between the central maxima and outer maxima/minima was measured using a pair of vernier calipers. 
As the tick marks for the maxima/minima were approximations of the centers of each peak, the measurements given by the calipers had an additional uncertainty of 0.1 cm.
This uncertainty, alongside difficulty measuring pencil markings contributed to overall error in the wavelength values.
\begin{center}
    \begin{tabular}{|c|c|c|c|c|c|c|}
        \hline
        & $\bm{x_{+1}}$ &$\bm{x_{-1}}$ & $\bm{x_{+2}}$ & $\bm{x_{-2}}$ & $\bm{x_{+3}}$ & $\bm{x_{-3}}$  \\ \hline
        $\bm{\theta}$ \textbf{(rad)} & 0.006 & 0.007 & 0.013 & 0.013 & 0.018 & 0.019\\
        $\bm{a}$ \textbf{(cm)}       & 0.011 & 0.009 & 0.010 & 0.009 & 0.007 & 0.007 \\ \hline
        $\bm{a}$ & \multicolumn{6}{|c|}{$0.0087\pm0.0006$ \textbf{ cm}}\\\hline
    \end{tabular} 
    \vspace{3mm}
    \\ Table 10: Hair Thickness Determination\\
    \vspace{5mm}
    \includegraphics*[]{Hair Thickness.jpg}
    \\Plot 2: Hair Thickness With Associated Uncertainty
\end{center}
Plot 2 shows the final value of the thickness of a hair as determined via Babinet's principle. 
Due to how fine hair is, alongside the wide variety of hair types, it has been difficult for researchers to determine an exact value for the thickness of a hair.
Thin hair can be at a minimum thickness of $15 \mu$m, while thicker hair can have a diameter of $181 \mu$m. 
The hair we measured in class had a diameter of $87 \mu$m, putting it towards the middle of the expected range. 
Our value was offset from the average accepted value by about $12 \mu$m, but still remained within the accepted range of values.
This estimation has similar sources of uncertainty to the wavelength determinations of Part A and B, as the same process was followed in order to collect data on the maxima/minima produced by the slit.
