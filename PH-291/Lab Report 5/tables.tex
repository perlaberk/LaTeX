\begin{center}
    \begin{tabular}{|c|c|c c|}
        \hline
        \textbf{Measurement} & $\bm{D\textbf{ (cm)}}$ & $\bm{x_{+1}}$\textbf{(cm)} & $\bm{x_{-1}}$\textbf{(cm)} \\ \hline
        1 & 14.95 & 13.900 & 13.360 \\ 
        2 & 14.80 & 13.430 & 13.100 \\ 
        3 & 14.75 & 13.300 & 12.942 \\ 
        4 & 14.75 & 13.250 & 13.236 \\  \hline \hline
        $\bm{\mu}$ & 14.81 & 13.470 & 13.160 \\  
        \textbf{Random Error} & 0.04 & 0.13 & 0.08 \\ 
        \textbf{Instrumental Error} & $\pm 0.05$ & \multicolumn{2}{c|}{$\pm 0.002$} \\ \hline
    \end{tabular}
    \vspace{3mm}
    \\Table 1: Interference due to a grating  \\
\end{center}
Here, $D$ is the distance between the grating and the screen. This was measured using a meter stick. 
The spacing between each slit is the reciprocal of the number of lines/mm for the grating. Our grating had 1000 lines/mm, which gave us $d=1*10^{-4}$ cm.
Using $D$ and the distances between the center and the principal maxima ($x_{+1}$ and  $x_{-1}$), we can calculate $\theta$, which allows us to calculate the wavelength of the laser.
When measuring the centers of each peak with vernier calipers, we found that there was an uncertainty of $\pm 0.1$ cm in addition to the instrumental error associated with the calipers.  
This affected our measurements for $x_{+1}$ and $x_{-1}$ and affected the calculated wavelength values.\\
We then calculated the wavelength by using a single slit. These set using a pair of vernier calipers, which have an instrumental error of 0.002 cm. Measurements were taken for slit widths of 0.2mm, 0.3mm, and 0.4mm.
\begin{center}
    \begin{tabular}{|c|c|c c c c|}
        \hline
        \textbf{Measurement} & $\bm{D\textbf{ (cm)}}$ & $\bm{x_{+1}}$\textbf{(cm)} & $\bm{x_{-1}}$\textbf{(cm)} & $\bm{x_{+2}}$\textbf{(cm)} & $\bm{x_{-2}}$\textbf{(cm)}\\ \hline
        1                     & 49.40& 0.130 & 0.156  & 0.320 & 0.330 \\ 
        2                     & 49.30& 0.136 & 0.158  & 0.298 & 0.320 \\ 
        3                     & 49.40& 0.158 & 0.156  & 0.320 & 0.346 \\ 
        4                     & 49.30& 0.138 & 0.160  & 0.318 & 0.354 \\ \hline \hline
        $\bm{\mu}$            & 49.35& 0.141 & 0.158  & 0.314 & 0.338 \\ 
        \textbf{Random Error} & 0.03 & 0.005 & 0.0008 & 0.005 & 0.007 \\ 
        \textbf{Instrumental Error} & $\pm 0.05$ & \multicolumn{4}{c|}{$\pm 0.002$} \\ \hline
    \end{tabular}
    \vspace{3mm}
    \\Table 2: 0.2mm Single Slit Diffraction\\
    \vspace{5mm}
\newpage
    \begin{tabular}{|c|c|c c c c|}
        \hline
        \textbf{Measurement} & $\bm{D\textbf{ (cm)}}$ & $\bm{x_{+1}}$\textbf{(cm)} & $\bm{x_{-1}}$\textbf{(cm)} & $\bm{x_{+2}}$\textbf{(cm)} & $\bm{x_{-2}}$\textbf{(cm)}\\ \hline
        1                     & 69.20& 0.110 & 0.118  & 0.246 & 0.244 \\ 
        2                     & 69.20& 0.100 & 0.120  & 0.260 & 0.254 \\ 
        3                     & 69.20& 0.108 & 0.138  & 0.262 & 0.268 \\ 
        4                     & 69.20& 0.102 & 0.136  & 0.258 & 0.240 \\ \hline \hline
        $\bm{\mu}$            & 69.20& 0.105 & 0.128  & 0.257 & 0.252 \\ 
        \textbf{Random Error} & 0.00 & 0.002 & 0.005  & 0.003 & 0.005 \\ 
        \textbf{Instrumental Error} & $\pm 0.05$ & \multicolumn{4}{c|}{$\pm 0.002$} \\ \hline
    \end{tabular}
    \vspace{3mm}
    \\Table 3: 0.3mm Single Slit Diffraction\\
    \vspace{5mm}
    \begin{tabular}{|c|c|c c c c|}
        \hline
        \textbf{Measurement} & $\bm{D\textbf{ (cm)}}$ & $\bm{x_{+1}}$\textbf{(cm)} & $\bm{x_{-1}}$\textbf{(cm)} & $\bm{x_{+2}}$\textbf{(cm)} & $\bm{x_{-2}}$\textbf{(cm)}\\ \hline
        1                     & 114.80& 0.198 & 0.180  & 0.410 & 0.430 \\ 
        2                     & 114.70& 0.260 & 0.190  & 0.420 & 0.432 \\ 
        3                     & 114.75& 0.265 & 0.198  & 0.420 & 0.454 \\ 
        4                     & 114.80& 0.196 & 0.194  & 0.410 & 0.420 \\ \hline \hline
        $\bm{\mu}$            & 114.76& 0.230 & 0.191  & 0.415 & 0.434 \\ 
        \textbf{Random Error} & 0.02  & 0.019 & 0.004  & 0.003 & 0.007 \\ 
        \textbf{Instrumental Error} & $\pm 0.05$ & \multicolumn{4}{c|}{$\pm 0.002$} \\ \hline
    \end{tabular}
    \vspace{3mm}
    \\Table 4: 0.4mm Single Slit Diffraction\\
\end{center}
Using each of the distance between the center and the peaks (measured with vernier calipers), we were able to calculate a value for the wavelength. 
We can then use this wavelength to find a thickness of a strand of hair using data collected from the diffraction produced by the hair.
\newpage
\begin{center}
    \begin{tabular}{|c|c|c c c c c c|}
        \hline
        \textbf{Measurement} & $\bm{D}$ & $\bm{x_{+1}}$ & $\bm{x_{-1}}$ & $\bm{x_{+2}}$ & $\bm{x_{-2}}$ & $\bm{x_{+3}}$ & $\bm{x_{-3}}$\\ \hline
        1                     & 56.70  & 0.340 & 0.410  & 0.746 & 0.770 & 1.040 & 1.070 \\ 
        2                     & 56.60  & 0.310 & 0.412  & 0.744 & 0.760 & 1.044 & 1.065\\ 
        3                     & 56.65  & 0.330 & 0.402  & 0.752 & 0.782 & 1.043 & 1.080\\ 
        4                     & 56.69  & 0.326 & 0.420  & 0.756 & 0.720 & 1.041 & 1.059\\ \hline \hline
        $\bm{\mu}$            & 56.66  & 0.327 & 0.411  & 0.750 & 0.758 & 1.042 & 1.069\\ 
        \textbf{Random Error} & 0.02   & 0.005 & 0.003  & 0.002 & 0.012 & 0.001 & 0.004\\ 
        \textbf{Instrumental Error} & $\pm 0.05$ & \multicolumn{6}{c|}{$\pm 0.002$} \\ \hline
    \end{tabular}
    \vspace{3mm}
    \\Table 5: Hair Diffraction (cm)\\

\end{center}
Using Babinet's principle we were able to calculate the thickness of a hair. 
The hair was estimated to be rectangular, making it have an equivalent diffraction pattern to a rectangular slit. 
The equation that was used to solve for the wavelength was rearranged to allow us to solve for the thickness of the hair, as detailed in the calculations section.
