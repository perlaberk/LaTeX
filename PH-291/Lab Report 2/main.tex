\NeedsTeXFormat{LaTeX2e}

\documentclass[12pt]{article}
\usepackage[letterpaper,portrait, margin=1in]{geometry}
\usepackage{booktabs, pgfplots, bm, multirow, amsmath}
\pgfplotsset{width=11cm,compat=1.9}
\renewcommand{\arraystretch}{1.5}
\usepackage[colorlinks=true, allcolors=blue]{hyperref}
\usepackage{indentfirst}



\begin{document}
\input{Lab 2 Cover Page.tex}
\newpage
\tableofcontents
\newpage
\section{Purpose}
In this lab, Pfund's Method and Snell's Law will be used to determine the index of refraction of an unknown liquid. 
Two methods will be used to find this index of refraction, the first utilizing Pfund's Method, while the second uses Snell's Law.
The index of refraction of the glass petri dish is determined via Pfund's Method. This then allows us to proceed with both methods. 
As light is refracted through a medium, a small circular region is observed, surrounded by a darker region with a radius that can be used to calculate the unknown index of refraction via Pfund's Law.
In the second method, the incident and reflected angles of the light are measured and plugged into Snell's Law. We can compare the two calculated values to determine the unknown substance.
\newpage
\section{Data}
\begin{center}
    \begin{table}[!ht]
        \centering
        \begin{tabular}{|l|c|}
        \hline
            \textbf{Measurement} & \textbf{Thickness (mm)}  \\ \hline
            1 & 1.695 \\ \hline
            2 & 1.995 \\ \hline
            3 & 1.990  \\ \hline
            4 & 2.095 \\ \hline
            5 & 1.930 \\ \hline
            6 & 1.995 \\ \hline \hline
            \textbf{Instrumental Error (mm)} & 0.005 \\ \hline
            \textbf{Random Error (mm)} & 0.055 \\ \hline
            \textbf{Thickness (mm)}   & $1.950\pm0.005$ \\ \hline
        \end{tabular}
        \caption{Petri Dish Thickness}
    \end{table}
    \begin{table}[!ht]
    \centering
    \begin{tabular}{|l|c|}
    \hline
        \textbf{Measurement} & \textbf{Diameter (mm)} \\ \hline
        1 & 7.98 \\ \hline
        2 & 8.86 \\ \hline
        3 & 8.24 \\ \hline
        4 & 7.60 \\ \hline
        5 & 8.16 \\ \hline
        6 & 7.66 \\ \hline \hline
        \textbf{Instrumental Error (mm)} & 0.02 \\ \hline
        \textbf{Random Error (mm)} & 0.19 \\ \hline
        \textbf{Diameter (mm)}   & $8.08\pm0.02$ \\ \hline
    \end{tabular}
    \caption{Ring Diameter Without Liquid}
    \end{table}
    \begin{table}[!ht]
        \centering
        \begin{tabular}{|l|c|}
        \hline
            \textbf{Measurement} & \textbf{Diameter (mm)} \\ \hline
            1 & 19.68 \\ \hline
            2 & 19.60 \\ \hline
            3 & 19.58 \\ \hline
            4 & 19.58 \\ \hline
            5 & 19.56 \\ \hline
            6 & 19.62 \\ \hline \hline
            \textbf{Instrumental Error (mm)} & 0.02 \\ \hline
            \textbf{Random Error (mm)} & 0. \\ \hline
            \textbf{Diameter (mm)}   & $7.46\pm0.02$ \\ \hline
        \end{tabular}
        \caption{Ring Diameter With Liquid}
        \end{table}
    \begin{table}[!ht]
    \centering
    \begin{tabular}{|l|l|l|}
    \hline
        \textbf{Measurement} & \textbf{Incident Angle} & \textbf{Refracted Angle} \\ \hline
        1 & 50 & 28 \\ \hline
        2 & 28 & 38 \\ \hline
        3 & 36 & 20 \\ \hline
        4 & 17 & 40 \\ \hline
        5 & 43 & 27 \\ \hline
        6 & 28 & 39 \\ \hline \hline
        \textbf{Instrumental Error (Degrees)} & \multicolumn{2}{|c|}{0.5} \\ \hline
    \end{tabular}
    \caption{Snell's Law}
    \end{table}
\end{center}
\newpage
\section{Calculations}
\[insert here\]

\end{document}