\NeedsTeXFormat{LaTeX2e}

\documentclass[12pt]{article}
\usepackage[letterpaper,portrait, margin=1in]{geometry}
\usepackage{booktabs, pgfplots, bm, multirow, amsmath, gensymb}
\pgfplotsset{width=11cm,compat=1.9}
\renewcommand{\arraystretch}{1.5}
\usepackage[colorlinks=true, allcolors=blue]{hyperref}
\usepackage{indentfirst}



\begin{document}
\begin{titlepage}

       %\vspace*{.5cm}
        \begin{center}
        \textbf{\huge PH-291 Physics Lab} \\ 
        \textbf{\Large Professor Corn-Agostini} \\ 
        \textbf{\Large Fall 2022} \\ 
        \vspace*{.5cm}  
        \textbf{\large Lab \# 2: Index of Refraction}
        \vspace{0.5cm}
        \end{center}
       
\noindent Your Name: Perla Berkovitz\\ \\
\noindent Your Lab Section:PH-291-E\\ \\
\noindent Your Lab Instructor: Professor Corn-Agostini\\ \\
\noindent Your Lab Partner's Name: Sharon Sitt\\ \\
\noindent Read and sign Academic Integrity Statement:\\

\noindent {\em I hereby attest that I have not given or received any unauthorized assistance on this assignment.}

    \begin{center}
    \line(1,0){300} \\
    Sign here
    \end{center}

\noindent\textbf{\large Grading Rubric} \\ \\
\renewcommand{\arraystretch}{1.5}
%\large
\begin{tabular}{|l|c|r|l|} 
\hline
 {\bf SECTION } & {\bf POINTS} & {\bf GRADE} & {\bf COMMENTS.................................}\\\hline 
Purpose & 1 & & \\\hline
Data & 3 & & \\\hline
Explanation of Errors & 3 & & \\\hline
Calculations  & 3 & & \\ \hline
Results & 2 & & \\\hline 
Conclusion & 1 & & \\ \hline
Answers & 2 & & \\\hline \hline 
{\em Total} & 15 & & \\ \hline
\end{tabular}

\end{titlepage} 

\newpage
\tableofcontents
\newpage
\section{Purpose}
In this lab, Pfund's Method and Snell's Law will be used to determine the index of refraction of an unknown liquid. 
Two methods will be used to find this index of refraction, the first utilizing Pfund's Method, while the second uses Snell's Law.
The index of refraction of the glass petri dish is determined via Pfund's Method. This then allows us to proceed with both methods. 
As light is refracted through a medium, a small circular region is observed, surrounded by a darker region with a radius that can be used to calculate the unknown index of refraction via Pfund's Law.
In the second method, the incident and reflected angles of the light are measured and plugged into Snell's Law. We can compare the two calculated values to determine the unknown substance.
\newpage
\section{Data}
\begin{center}
    \begin{minipage}{.5\linewidth}
        \centering
        \begin{tabular}{|c | c|}
            \hline
            \textbf{Measurement} & \textbf{Thickness (mm)}  \\ \hline
            1 & 1.695 \\ 
            2 & 1.995 \\ 
            3 & 1.990  \\ 
            4 & 2.095 \\ 
            5 & 1.930 \\ 
            6 & 1.995 \\ \hline
            \multicolumn{2}{|c|}{\textbf{Instrumental Error:} 0.005 mm} \\
            \multicolumn{2}{|c|}{\textbf{Random Error:} 0.055 mm} \\
            \multicolumn{2}{|c|}{\textbf{Thickness:} $1.950\pm0.055$ mm} \\ \hline
        \end{tabular}
        \vspace{3mm}
        \\Table 1: Petri Dish Thickness
        \vspace{10mm}
    \end{minipage}%
    \begin{minipage}{.5\linewidth}
        \centering
        \begin{tabular}{|c | c|}
            \hline
            \textbf{Measurement} & \textbf{Thickness (mm)}  \\ \hline
            1 & 7.98 \\ 
            2 & 8.86 \\ 
            3 & 8.24  \\ 
            4 & 7.60 \\ 
            5 & 8.16 \\ 
            6 & 7.66 \\ \hline
            \multicolumn{2}{|c|}{\textbf{Instrumental Error:} 0.02 mm} \\
            \multicolumn{2}{|c|}{\textbf{Random Error:} 0.19 mm} \\
            \multicolumn{2}{|c|}{\textbf{Thickness:} $8.08\pm0.19$ mm} \\ \hline
        \end{tabular}
        \vspace{3mm}
        \\Table 2: Ring Diameter Without Liquid
        \vspace{10mm}
    \end{minipage} 
    \begin{tabular}{|c|c|}
        \hline
        \textbf{Measurement} & \textbf{Diameter (mm)} \\ \hline
        1 & 19.68 \\ 
        2 & 19.60 \\ 
        3 & 19.58 \\ 
        4 & 19.58 \\ 
        5 & 19.56 \\ 
        6 & 19.62 \\ \hline
        \multicolumn{2}{|c|}{\textbf{Instrumental Error:} 0.02 mm} \\
        \multicolumn{2}{|c|}{\textbf{Random Error:} 0.017 mm} \\
        \multicolumn{2}{|c|}{\textbf{Thickness:} $19.60\pm0.02$ mm} \\ 
        \hline
    \end{tabular}
    \vspace{3mm}
    \\Table 3: Ring Diameter With Liquid
    \begin{tabular}{|c|c c|}
    \hline
        \textbf{Measurement} & \textbf{Incident Angle} & \textbf{Refracted Angle} \\ \hline
        1 & $50.0\degree$ & $28.0\degree$ \\ 
        2 & $28.0\degree$ & $38.0\degree$ \\ 
        3 & $36.0\degree$ & $20.0\degree$ \\ 
        4 & $17.0\degree$ & $40.0\degree$ \\ 
        5 & $43.0\degree$ & $27.0\degree$ \\ 
        6 & $28.0\degree$ & $39.0\degree$ \\ \hline
        \multicolumn{3}{|c|}{\textbf{Instrumental Error:} $0.5\degree$} \\ \hline
    \end{tabular}
    \vspace{3mm}
    \\Table 4: Snell's Law
\end{center}
\newpage
\section{Calculations}
\subsection*{Pfund's Method}
\noindent \textbf{1. Index of Refraction $\bm{({n_{\textbf{glass}}})}$}\[n_{\textnormal{glass}}=\frac{\sqrt{d^2+16t^2}}{d}\]
\subsubsection*{Error Propagation:}
    \noindent\textbf{1.1 Partial Derivative of Eq. 1 w.r.t $\bm d$}\[\frac{\partial n_{\textnormal{glass}}}{\partial d}=-\dfrac{16t^2}{d^2\sqrt{d^2+16t^2}}\]
    \textbf{1.2 Partial Derivative of Eq. 1 w.r.t $\bm t$} \[\frac{\partial n_{\textnormal{glass}}}{\partial t}=\dfrac{16t}{d\sqrt{16t^2+d^2}}\]
    \textbf{1.3 Total Error Associated with the Index of Refraction $\bm{({n_{\textbf{glass}}})}$}\[\delta n_{\textnormal{glass}} =\sqrt{\left(\frac{\partial n_{\textnormal{glass}}}{\partial t}\delta t\right)^2+\left(\frac{\partial n_{\textnormal{glass}}}{\partial d}\delta d\right)^2}\]
\\\textbf{2. Index of Refraction $\bm{({n_{\textbf{liquid}}})}$}\[n_{\textnormal{liquid}}=\frac{n_{\textnormal{glass}}d}{\sqrt{d^2+16t^2}}\]
\subsubsection*{Error Propagation:}
    \noindent\textbf{2.1 Partial Derivative of Eq. 2 w.r.t $\bm d$}\[\frac{\partial n_{\textnormal{liquid}}}{\partial d}=\dfrac{16n_\textnormal{glass}t^2}{\left(d^2+16t^2\right)^\frac{3}{2}}\]
    \textbf{2.2 Partial Derivative of Eq. 2 w.r.t $\bm t$} \[\frac{\partial n_{\textnormal{liquid}}}{\partial t}=-\dfrac{16dn_\text{glass}t}{\left(16t^2+d^2\right)^\frac{3}{2}}\]
    \textbf{2.3 Partial Derivative of Eq. 2 w.r.t $\bm n_\textbf{glass}$}\[\frac{\partial n_{\textnormal{liquid}}}{\partial n_\text{glass}}=\dfrac{d}{\sqrt{16t^2+d^2}}\]
    \textbf{2.4 Total Error Associated with the Index of Refraction $\bm{({n_{\textbf{liquid}}})}$}\[\delta n_{\textnormal{liquid}} =\sqrt{\left(\frac{\partial n_{\textnormal{liquid}}}{\partial t}\delta t\right)^2+\left(\frac{\partial n_{\textnormal{liquid}}}{\partial d}\delta d\right)^2+\left(\frac{\partial n_{\textnormal{liquid}}}{\partial n_{\text{glass}}}\delta n_{\text{glass}}\right)^2}\]
\subsection*{Snell's Law}
\noindent\textbf{3. The Law of Refraction (Snell's Law)}\[n_1\sin{\theta_1}=n_2\sin{\theta_2}\]
\subsubsection*{Error Propagation}
\noindent\textbf{3.1 Partial Derivative of Eq. 3 w.r.t $\bm \theta_1$}
\end{document}