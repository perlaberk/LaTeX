\NeedsTeXFormat{LaTeX2e}

\documentclass[12pt]{article}
\usepackage[letterpaper,portrait, margin=1in]{geometry}
\usepackage{booktabs, pgfplots, bm, multirow, amsmath, wrapfig, gensymb, graphicx}
\pgfplotsset{width=11cm,compat=1.9}
\renewcommand{\arraystretch}{1.5}
\usepackage[colorlinks=true, allcolors=blue]{hyperref}
\usepackage{indentfirst}
\usepackage{caption}



\begin{document}
\section*{Selfie Improvement}
    \begin{minipage}{.5\textwidth}
      \centering
      \includegraphics[width=\linewidth]{bad selfie.png}
      \captionof{figure}{First Selfie Taken}
      \label{fig:test1}
      \vspace{5mm}
    \end{minipage}%
    \begin{minipage}{.5\textwidth}
      \centering
      \includegraphics[width=\linewidth]{good selfie.png}
      \captionof{figure}{Improved Selfie}
      \label{fig:test2}
      \vspace{5mm}
    \end{minipage}
    
    When the first selfie was taken, I had positioned myself with the light behind me. This caused the camera to not have enough light on my face to capture a detailed image. In the improved selfie, there was better lighting, which allowed for a better image to be captured.
\section*{Physical Phenomenon}
\begin{center}
\includegraphics[scale=0.2]{physical principle.png}
\captionof{figure}{Physical Phenomenon}
\end{center}
  For my physical phenomenon, I chose motion and relative velocity. In the image, the motion streaks on the right hand side are a result of the cars traveling past at some velocity, while the camera remained stationary. Like the camera, other cars in the image and the surroundings remain stationary which have velocity = 0.
\section*{Alternative Developer Recipe}
\begin{center}
    \includegraphics[scale=0.2]{alternate developer.png}
    \captionof{figure}{Alternate Developer}
    \end{center}
\noindent200 mL Water\\
1 g Vitamin C\\
1.2 g Instant Coffee\\
10 g Washing Soda\\
\end{document}